\documentclass[]{article}
\usepackage{lmodern}
\usepackage{amssymb,amsmath}
\usepackage{ifxetex,ifluatex}
\usepackage{fixltx2e} % provides \textsubscript
\ifnum 0\ifxetex 1\fi\ifluatex 1\fi=0 % if pdftex
  \usepackage[T1]{fontenc}
  \usepackage[utf8]{inputenc}
\else % if luatex or xelatex
  \ifxetex
    \usepackage{mathspec}
  \else
    \usepackage{fontspec}
  \fi
  \defaultfontfeatures{Mapping=tex-text,Scale=MatchLowercase}
  \newcommand{\euro}{€}
\fi
% use upquote if available, for straight quotes in verbatim environments
\IfFileExists{upquote.sty}{\usepackage{upquote}}{}
% use microtype if available
\IfFileExists{microtype.sty}{%
\usepackage{microtype}
\UseMicrotypeSet[protrusion]{basicmath} % disable protrusion for tt fonts
}{}
\makeatletter
\@ifpackageloaded{hyperref}{}{%
\ifxetex
  \usepackage[setpagesize=false, % page size defined by xetex
              unicode=false, % unicode breaks when used with xetex
              xetex]{hyperref}
\else
  \usepackage[unicode=true]{hyperref}
\fi
}
\@ifpackageloaded{color}{
    \PassOptionsToPackage{usenames,dvipsnames}{color}
}{%
    \usepackage[usenames,dvipsnames]{color}
}
\makeatother
\hypersetup{breaklinks=true,
            bookmarks=true,
            pdfauthor={},
            pdftitle={How to write a consumer for a typescript library},
            colorlinks=true,
            citecolor=blue,
            urlcolor=blue,
            linkcolor=magenta,
            pdfborder={0 0 0}
            }
\urlstyle{same}  % don't use monospace font for urls
\setlength{\parindent}{0pt}
\setlength{\parskip}{6pt plus 2pt minus 1pt}
\setlength{\emergencystretch}{3em}  % prevent overfull lines
\providecommand{\tightlist}{%
  \setlength{\itemsep}{0pt}\setlength{\parskip}{0pt}}
\setcounter{secnumdepth}{0}

\title{How to write a consumer for a typescript library}
\date{}

% Redefines (sub)paragraphs to behave more like sections
\ifx\paragraph\undefined\else
\let\oldparagraph\paragraph
\renewcommand{\paragraph}[1]{\oldparagraph{#1}\mbox{}}
\fi
\ifx\subparagraph\undefined\else
\let\oldsubparagraph\subparagraph
\renewcommand{\subparagraph}[1]{\oldsubparagraph{#1}\mbox{}}
\fi

\begin{document}
\maketitle

\{\{\textgreater{} ./components/analytics.html \}\} \{\{\textgreater{}
./components/header/header.html\}\}

\section{Writing a command-line executable with
TypeScript}\label{writing-a-command-line-executable-with-typescript}

On \href{/}{the main tutorial} we've setup the library
\texttt{\textquotesingle{}hwrld\textquotesingle{}}. Now we're going to
learn how to make a library available as a system command. Actually,
once you've completed the main tutorial, this becomes really easy. Just
two simple steps!

\subsection{\texorpdfstring{{Step1:} Add execution
instructions}{Step1: Add execution instructions}}\label{step1-add-execution-instructions}

On top of your executable files (the main files), add the following
line:

\begin{verbatim}
#!/usr/bin/env node
\end{verbatim}

This makes sure the system understands how to execute the compiled
javascript file, by instructing it to interpret it with \texttt{node}.

\subsection{\texorpdfstring{{Step 2:} Modify the
package.json}{Step 2: Modify the package.json}}\label{step-2-modify-the-package.json}

The \texttt{package.json} just needs to be modified a bit, so it looks
like this:

typescript-library/package.json

\begin{verbatim}
{
  "name": "hwrld",
  "version": "1.0.0",
  "description": "Can log \"hello world\" and \"goodbye world\" to the console!",
  "main": "dist/index.js",
  "bin": {
      "hwrld": "dist/index.js"
  },
  "types": "dist/index.d.ts"
}
\end{verbatim}

And voilà! Once you publish this, you will be able to install you're
package globally on a machine using:

\begin{verbatim}
sudo npm install -g hwrld
\end{verbatim}

Of course for the \texttt{hwrld} package this is pretty useless. A
really simple yet instructive example I've written is a controller for
the \texttt{mpc} music player. You can find the code here:
\url{https://github.com/bersling/mpc-control}.

By the way, there wasn't really anything typescripty about this
tutorial, it holds just as well for plain javascript node modules!

You can now \ldots{}

\begin{itemize}
\tightlist
\item
  \href{}{\ldots{} go back to the main tutorial}
\item
  \href{./unit-testing}{\ldots{} read about unit testing}
\item
  \href{./local-consumer}{\ldots{} read about how to build a local
  consumer}
\end{itemize}

\{\{\textgreater{} ./components/article/article-footer.html\}\}

\end{document}
